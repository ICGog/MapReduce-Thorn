\section{Changes to Syntax}

\begin{enumerate}

\item \label{syntax:pure} Purity is currently marked by \kw{:pure}, which was going to be an instance of
some general syntax, but is not. Make it some other kind of modifier, like a
prefix, if we keep it at all. 

\item \label{syntax:assignment} We could allow `\kw{var} x = 4' as a synonym for `\kw{var} x \kw{:=} 4'.  It's not clear that
this is a good idea, since using `=' for assignments may be confusing.

\item  \label{syntax:semicolons as separators} The `;' after `2' should be optional, but is not:
\begin{quote}
\begin{lstlisting}
fun f(1) = 2;
|     f(_) = 3;
\end{lstlisting}
\end{quote}

\item \label{syntax:modifiers} \kw{from}, \kw{envelope}, and \kw{prio} clauses on method and function definitions
should be allowed in arbitrary order. 

\item \label{syntax:super} The constructor supercall should allow \kw{new}$@$\kw{super}(...) as well as \kw{new}$@$A(...).

\item \label{syntax:import own} Remove \kw{import own}.

\item \label{syntax:module:private/public} Right now, public/private statements in a module are separate statements: 
\begin{quote}
\begin{lstlisting}
module Moo {
	var a; var b; var c;
	private a; private b;
}
\end{lstlisting}
\end{quote}
Having \kw{private} be a modifier on declarations (and getting rid of \kw{public})
would probably be better style. Failing which, at least allow:  \kw{private} a,b;

\item \label{syntax:expressions as statements} Currently a few choices of statement can be put in parens to make an
expression: (\kw{if} (true) 3; \kw{else} 4;). Probably all of them should.

\item \label{syntax:records} \{: ... :\} is pretty unsatisfying for something as fundamental as records. 
They are far more important than the two-char syntax indicates.
Go for  \{ x : 1, y : 2 \} if possible.  Another proposal is to use `\kw{new}  \{ x : 1; y : 2 \}'.

\item \label{syntax:.nice(x)} Remove the `.nice(x)' pattern. 
The currently supported alternative `it.nice $\sim$ +x' isn't much wordier, and is
clearer and more general.  Maybe also eliminate the `.nice()' pattern, which abbreviates
`it.nice $\sim$ !(null $||$ false)'.

\item \label{syntax:case} Change the syntax currently described as the nonterminal 'cases': 
\begin{quote}
\begin{lstlisting}
try {
	x := 1;
} catch {
	``Doom!'' => {print(``doom'');}
	| ``Mood?'' => {} 
}
\end{lstlisting}
\end{quote}
The `$|$' separating the two cases is too small visually to separate chunks of
code. Allow something else as an option: perhaps `\kw{case}'.

\item \label{syntax:table decls}The syntax for table column declarations is awkward:
\begin{lstlisting}
t2 = table(x,y){var fg; val bg; zog : int;}; 
\end{lstlisting}
The current syntax is intended to parallel var/val decls, which it
does, but that's not so nice.  It would be better to turn the semicolons into commas, and have the last one be
optional:
\begin{lstlisting}
t2 = table(x,y){var fg, val bg, zog : int}
\end{lstlisting}
Ditto for ords.

\item \label{syntax:type constraints} Replace the type constraint operation `:'  with `::',
which is probably essential to revising record syntax and perhaps conditional
expressions. 

\item \label{syntax:conditional expressions} Could we get away with using C-style conditional expressions, i.e. `A ? B : C'? Originally we didn't do that
because the `:' is being used for other things, but we may now have it clear
enough.

Failing which---could `if A then B else C' be an expression with ANTLR? It
didn't work in JavaCC, for LL reasons.

\item \label{syntax:queries} The \% sign in queries is a problem.  Can we do the following:

\begin{description}

\item[Vars] \hfill \\ Most query controls don't need adaptation. The one that does is 'var'
\begin{center}`\lstinline{var i := E %then1 F}' $\rightarrow$ `\lstinline{var i := E ,, F}'\end{center}
(Note---consistent with the use of ,, in groups, later on.)
\begin{center}`\lstinline{var i := E %then0 F}' $\rightarrow$ `\lstinline{var i := ,, E ,, F}'\end{center}
This is not ideal.

\item[List Comprehension] \hfill \\ 
$[$ Exp ...? $|$ QueryControls $]$---that is, using the presence of a `$|$' early in the list to disambiguate
from list constructors?

\item[Quantifiers]  \hfill \\ 
Using \kw{count}, \kw{all}, \kw{exists} instead of \kw{\%count}, \kw{\%all}, \kw{\%exists} is
probably good enough. There is some issue about claiming a popular
identifier like `count' as a reserved word.

\item[First]  \hfill \\ 
Use \kw{first} instead of \kw{\%first}/\kw{\%find}.
Use \kw{find} as the statement-level variant (drop \kw{first} as synonym).
Use \kw{else} for \kw{\%none}.

\item[After]  \hfill \\  
Use \kw{after} instead of \kw{\%after}.

\item[Table Comprehension]  \hfill \\  
Can the `$|$' in the table constructor work here too? And the `=' in the key
fields?
\begin{lstlisting}
{table(x=i){xsq = i*i | for i <- 1..100} 
\end{lstlisting}
vs a table constructor: 
\begin{lstlisting}
table(x){xsq}
\end{lstlisting}

\item[Sorting]  \hfill \\   
Use \kw{sort} instead of \kw{\%sort}.
Use \kw{by} instead of \kw{\%<}.
Use \kw{yb} instead of \kw{\%>}.
Or maybe \kw{by decreasing} for a more cobolly effect?

\item[Group queries]  \hfill \\   
Use \kw{count} instead of \kw{\%count}  (a previously introduced keyword);
`\kw{E ...}' instead of `\kw{\%list E}'; 
 `\kw{... E}' instead of `\kw{\%rev E}'; and
 `E ,, F ,, G' instead of `\kw{\%first E \%then F \%after G}'.

So, oldstyle: 
\begin{lstlisting}
%group(prime? = n.prime?) {
num = %count;
them = %list n;
meht = %rev n;
sum = %first n %then sum+n;
sumAsString = %first n %then 
  sum+n %after sum.str;
| for n <- 2 .. 100  }
\end{lstlisting}
would be, newstyle: 
\begin{lstlisting}
group(prime? = n.prime?) {
num = count,
them = n ...,
meht = ... n,
sum = n ,, sum+n,
sumAsString = n ,, sum+n ,, sum.str
| for n <- 2..100 }
\end{lstlisting}

\end{description}

\item \label{syntax:timeout} Timeout clauses in \kw{recv} statements should have the same syntax as other clauses, mutatis mutandis.    

Currently: 
\begin{lstlisting}
  msg = (recv{
    x => x
    timeout(1000) { "timed out";}
  });
\end{lstlisting}

The timeout clause requires braces and no $=>$; it should look more like a regular clause.

\item \label{syntax:trailing semi-colons} We should allow 
\begin{lstlisting}
def glurd() {frell();};
\end{lstlisting}

that is, def's (and many other things) should be allowed to have trailing semicolons.

\item \label{syntax:spawn} The \kw{body} block in a \kw{spawn} should just be the body of the spawn. �Things like initially, reinit, before, after should just be methods of the component and not baked in. �Spawn should be similar to \kw{new} and just take a class name (a subclass of Component providing the appropriate methods) and arguments to pass to the constructor. �\kw{spawn} \{ ... \} is equivalent to \kw{spawn} $<$anon subclass of Component$>$. �One should be able to do \kw{new} \{ ... \} to create an anonymous subclass of Any.

\item \label{syntax:implicit initializer} The body of a class declaration should allow expressions invoked before the constructor---like an initializer block in Java.

\item \label{syntax:ellipses in patterns} Replace the `...' patterns with the `x::xs' syntax.

\item \label{syntax:variables in patterns} \$(k) in patterns should be writable as \$k.

\item \label{syntax:newline as terminator} Make $\backslash$n a synonym for `;'. �The semicolon should be used as a separator consistently.

\item \label{syntax:singleton objects} Remove singleton objects. �Should be using top-level module definitions instead.

\item \label{syntax:function def matching} Remove matches baked into function definitions. �Explicitly write:
\begin{lstlisting}�
fun foo(x) = match(x) { ... }
\end{lstlisting}
\end{enumerate}
