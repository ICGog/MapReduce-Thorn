\section{Resolved Syntax Changes}

\newlength{\itemboxwidth}
\newcommand{\itembox}[2]{%
\settowidth{\itemboxwidth}{#1}%
\parbox{\itemboxwidth}{\hfill#2}%
}
\newcommand{\yes}{\ensuremath{\checkmark}}
\newcommand{\no}{\ensuremath{\times}}
\newcommand{\maybe}{?}
\newcommand{\accept}[1]{\item[\yes\itembox{11.}{\ref{#1}.}]}
\newcommand{\reject}[1]{\item[\no\itembox{11.}{\ref{#1}.}]}
\newcommand{\pending}[1]{\item[\maybe\itembox{11.}{\ref{#1}.}]}

Legend: \yes\ marks adopted changes, \no\ marks rejected changes, and \maybe\ marks proposals that need further discussion.

\begin{enumerate}

\accept{syntax:pure} Purity is marked by the prefix modifier \kw{pure}. 

\reject{syntax:assignment}  Do not allow `=' to be used for assignments.

\accept{syntax:semicolons as separators} \hypertarget{resolved:semicolons as separators}{}Semicolons should be treated as (optional) separators rather than terminators.

\accept{syntax:modifiers} \kw{checked}, \kw{from}, \kw{envelope}, and \kw{prio} clauses on method and function definitions
are allowed in arbitrary order. 

\accept{syntax:super} Adopt the proposed change but use the keyword \kw{super} instead of \kw{new}.  That is, the constructor supercall should allow \kw{super}(...) as well as \kw{super}$@$A(...).

\accept{syntax:import own} Remove \kw{import own} in its current form.  However, we may re-introduce the functionality of \kw{import own} when we revise the module system.

\accept{syntax:module:private/public} \kw{private} must be a modifier on a member declaration. A member may not be marked as \kw{private} in a separate statement, after it has been declared.  We must allow \kw{public} to be used in a separate statement as a way to re-export specific members of imported modules. 

\accept{syntax:expressions as statements} \hypertarget{resolved:expressions as statements}{} Every Thorn statement should be usable as an expression.  Braces and parentheses can be used interchangeably to delimit expressions and eliminate ambiguity.  Braces introduce a new name scope; parentheses don't.  

\accept{syntax:records} Record syntax is now 
\begin{lstlisting}
<x = 1, y = 2> 
\end{lstlisting}

\accept{syntax:.nice(x)} Remove the `.nice(x)' pattern.
% The `.nice(x)' pattern can be useful, e.g.:
%\begin{lstlisting}
%[.int(arg1), .int(arg2)] = argv();
%\end{lstlisting}


\reject{syntax:case} Allow the following alternative syntax for cases in all matching contexts (e.g., match statement, receive statement, etc.): 
\begin{quote}
\begin{lstlisting}
try {
	x := 1;
} catch {
	case ``Doom!'' => {print(``doom'');}
	case ``Mood?'' => {} 
}
\end{lstlisting}
\end{quote}

\accept{syntax:table decls} Replace semicolons with commas and braces with angle brackets in table, ord and map declarations.  The keyword \kw{val} should be disallowed (it is the default), and the last comma should be optional.  For example:
\begin{lstlisting}
t2 = table(x,y)<var fg, bg, zog : int>
\end{lstlisting}


\reject{syntax:type constraints} Retain `:' for the type constraint operation. 

\reject{syntax:conditional expressions} Since if-statements can be used as expressions (see  \hyperlink{resolved:expressions as statements}{Resolved Changes \getrefnumber{syntax:expressions as statements}}), we don't need  C-style conditional expressions.


\pending{syntax:queries} Eliminate \% as a prefix to query keywords.  Adopt the current proposal except in the following cases:

\begin{description}

\item[Vars] \hfill \\ Query control in 'var' needs adaptation, but the current proposal to use double commas (instead of \kw{\%then1} and \kw{\%then0}) does not look nice.


\item[Quantifiers]  \hfill \\ 
Consider using \kw{cnt} or \# to replace \%\kw{count}.  Rethink the use of braces vs. parentheses vs. angle brackets to ensure consistency with other constructs.


\item[Table Comprehension]  \hfill \\  
Clarify the current proposal.

\item[Sorting]  \hfill \\   
Use \kw{inc} or \kw{increasing} instead of \kw{\%<}.
Use \kw{dec} or \kw{decreasing} instead of \kw{\%>}.


\item[Group queries]  \hfill \\   
Adopt the proposal except for the use of double commas `,,' to replace \%\kw{then} and \%\kw{after}.

\end{description}


\accept{syntax:timeout} Timeout clauses in \kw{recv} statements should have the same syntax as other clauses, e.g.: 
\begin{lstlisting}
  msg = (recv{
   x => { x; }
   | timeout 1000 =>  { "timed out"; }
  });
\end{lstlisting}


\accept{syntax:trailing semi-colons} See \hyperlink{resolved:semicolons as separators}{Resolved Changes \getrefnumber{syntax:semicolons as separators}}.

\pending{syntax:spawn}  Parts of this proposal may be applicable if we allow spawning of arbitrary code.


\reject{syntax:implicit initializer}  Do not allow initializer blocks in classes.
%If we allow initializer blocks in classes, they should have a keyword. We even have an \kw{init} keyword already in the language which we can reuse here: 
%\begin{lstlisting}
%var SnargCounter := 0;
%class Snarg{
%  val n;
%    init{
%      SnargCounter += 1; 
%      n = SnargCounter;
%    }
%}
%\end{lstlisting}

\reject{syntax:ellipses in patterns} Retain the functionality of the `...' patterns since they are strictly more expressive than `x::xs'.  However, replace the `...' syntax for these patterns with double dots (`..').

\accept{syntax:variables in patterns} \$(k) in patterns should be writable as \$k, where k is a simple identifier (alphanumeric characters only).  \$(e), where e is an expression, should be allowed in both strings and patterns.

\accept{syntax:newline as terminator} If $\backslash$n is allowed as a synonym for the semicolon, we need conventions to deal with incomplete expressions broken up over multiple lines.  For example
\begin{lstlisting}
e1 
+e2
\end{lstlisting}
One solution is to use a continuation character (e.g., '$\backslash$') to break up expressions over multiple lines.  For consistency, the same solution should be used for strings that are broken over multiple lines.

\reject{syntax:singleton objects} Not applicable to the Fisher version of Thorn.

\reject{syntax:function def matching} Keep the matching in function definitions. � 
 
%\item Eliminate triple dots completely, e.g., in list patterns.
\end{enumerate}


%\section{Pending Syntax Changes}
%\begin{enumerate}

%\item Is every Thorn expression  usable as a statement?

%\item Braces are required in the following constructs (in a non-expression contexts).  Should we revisit these?
%\begin{itemize}\renewcommand{\labelitemi}{--}

%\item Table-style queries.
%\item \kw{spawn} and stuff that goes inside of \kw{spawn} (e.g., \kw{body})

%\end{itemize}

%\item Can we use '*' to mean an arbitrary number of things (instead of '..')?

%\end{enumerate}
